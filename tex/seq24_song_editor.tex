%-------------------------------------------------------------------------------
% seq24_song_editor
%-------------------------------------------------------------------------------
%
% \file        seq24_song_editor.tex
% \library     Documents
% \author      Chris Ahlstrom
% \date        2015-08-31
% \update      2015-09-01
% \version     $Revision$
% \license     $XPC_GPL_LICENSE$
%
%     Provides the concepts.
%
%-------------------------------------------------------------------------------

\section{Song Editor}
\label{sec:seq24_song_editor}

   The \textsl{Sequencer24 Song Editor} is used to combine all of the patterns
   into a complete tune.  It works by showing one row per
   pattern/loop/sequence in numbered columns, and the placement of each
   pattern at various musical bars in the song.

   In \textsl{Sequencer24} parlance, the Song Editor creates a
   \textsl{performance}.

\begin{figure}[H]
   \centering 
   \includegraphics[scale=0.75]{song-editor/song-editor-window.png}
   \caption{Song Editor Window}
   \label{fig:song_editor_window}
\end{figure}

   This dialog is not too complex, but
   for exposition, we break it into a top panel and the rest of the window.

\subsection{Song Editor / Top Panel}
\label{subsec:seq24_song_editor_top}

   The top panel provides quick access to song-playback actions and
   configuration.

\begin{figure}[H]
   \centering 
   \includegraphics[scale=0.75]{song-editor/song-editor-top-panel-items.png}
   \caption{Song Editor / Top Panel Items}
   \label{fig:song_editor_top_panel_items}
\end{figure}

   \begin{enumber}
      \item \textbf{Stop}
      \item \textbf{Play}
      \item \textbf{Loop}
      \item \textbf{Beats Per Bar}
      \item \textbf{Beat Unit}
      \item \textbf{Grid Snap}
      \item \textbf{Undo}
      \item \textbf{Collapse}
      \item \textbf{Expand}
      \item \textbf{Expand and copy}
   \end{enumber}

   \setcounter{ItemCounter}{0}      % Reset the ItemCounter for this list.

   \itempar{Stop}{song editor!stop}
   Stops the playback of the song.
   \index{keys!esc (stop)}
   The keystroke for stopping playback is the 'Escape' character.
   It can be configured to be another character (such as 'Space', which
   would make the space-bar toggle the playback status.

   \itempar{Play}{song editor!play}
   \index{L marker}
   Starts the playback of the song, starting at the \textbf{L marker}.
   \index{keys!space (play)}
   The keystroke for starting playback is the 'Space' character.

   \itempar{Loop}{song editor!loop}
   \index{L marker}
   \index{R marker}
   Play the song, looped between the \textbf{L marker} and the
   \textbf{R marker}.
   This button is a state button, and its appearance indicates when it is
   depressed, and thus active.
   If this button is deactivated during playback, then playback will
   continue past the \textbf{R marker}.

   \itempar{Beats Per Bar}{song editor!beats/bar}
   Part of the time signature, and specifies the number of beat units per bar.
   The possible values range from 1 to 16.

   \itempar{Beat Unit}{song editor!beat unit}
   Part of the time signature, and specifies the size of the beat unit:
   1 for whole notes; 2 for half notes; 4 for quarter notes; 8 for eight notes;
   and 16 for sixteenth notes.

   \itempar{Grid Snap}{song editor!grid snap}
   Grid snap selects where the patterns will be drawn.
   Unlike the \textbf{Grid Snap} of the Pattern Editor, the units
   of the Song Editor snap value are in fractions of a measure length.
   The following values are supported:
   1/1, 1/2, 1/4, 1/8, 1/16, and 1/32.

   \itempar{Undo}{song editor!undo}
   The Undo button will roll back the last change in the layout of a
   pattern.  Each time it is clicked, the most recent change will be undone.
   It will roll back one change each time it is pressed.
   It is not certain what the undo limit is, however.
   There is no Redo button in the Song Editor.

   \itempar{Collapse}{song editor!collapse}
   This button collapses the song between the \textbf{L marker} and the
   \textbf{R marker}.
   What this means is that, if there is song material (patterns) before the
   \textbf{L marker} and after the \textbf{R marker},
   and the \textbf{Collapse} button is
   pressed, any song material between the L and R markers is wiped out, and
   the song material after the \textbf{R marker} is moved leftward to
   the \textbf{L marker}.

   Collapsing occurs in all tracks present in the Song Editor.

   \itempar{Expand}{song editor!expand}
   This button expands the song between the
   \textbf{L marker} and the \textbf{R marker}.
   It inserts blank space between these markers, moving the song material
   that is after the \textbf{R marker}
   to the right by the duration of the blank space.

   Expansion occurs in all tracks present in the Song Editor.

   \itempar{Expand and copy}{song editor!expand and copy}
   This button expands the song between the \textbf{L marker} and the
   \textbf{R marker} much like the \textbf{Expand} button.
   However, it also copies the original data that is present after the
   \textbf{R marker}, and pastes it into the newly-available space between
   the L and R markers.

\subsection{Song Editor / Arrangement Panel}
\label{subsec:seq24_song_editor_arrangement_panel}

   The arrangement panel is the middle section shown in
   \figureref{fig:song_editor_window}.  It is also known as the
   "piano roll" of the song editor. Here, we zero in on its many
   features.

   The following figure is taken from a conventional MIDI file, imported,
   with a few long tracks, rather than a large number of smaller patterns.
   In other words, the patterns used here are very long, and used only once
   in the song.
   
   We might need to provide an example that shows off \textsl{Sequencer24}'s
   pattern features better, at some point.

   Please note that, if playback is started with the Song Editor as the
   active window, then the pattern boxes in the patterns panel will
   show as armed/unarmed (unmuted/muted) depending upon whether or not the
   pattern is shown as playing (or not) at the current playback position in
   the Song Editor piano roll.

\begin{figure}[H]
   \centering 
   \includegraphics[scale=0.75]{song-editor/song-editor-window-full-items.png}
   \caption{Song Editor Arrangement Panel, Annotated}
   \label{fig:song_editor_window_full_items}
\end{figure}

   \index{measures ruler}
   It consists of a \textsl{measures ruler} (bar indicator) at the top, a
   numbered patterns column at the left with a muting indicator, and the
   grid or roll section.  There are a lot of hidden details in the
   arrangement panel, as the figure shows.  Here are the main sections we
   will deal with:

   \begin{enumber}
      \item \textbf{Patterns Column}
      \item \textbf{Piano Roll}
      \item \textbf{Measures Ruler}
   \end{enumber}

   These items are discussed in the following sections.

\subsubsection{Song Editor / Arrangement Panel / Patterns Column}
\label{subsubsec:seq24_song_editor_arrangement_panel_patterns_column}

   Here are the items to note in the patterns column:

   \begin{enumber}
      \item \textbf{Number}.
         Not yet sure what the number on the left means.
         The number of the screen set?
      \item \textbf{Title}.
         \index{pattern!title}
         \index{pattern!name}
         The title is the name of the pattern, for easy reference.
      \item \textbf{Channel}.
         \index{pattern!channel}
         The channel number appears at the right of the title.
      \item \textbf{Buss-Channel}.
         \index{pattern!span}
         This pair of numbers shows the MIDI buss number used in the pattern and
         the channel used for the pattern.
      \item \textbf{Beat/Measure}.
         \index{pattern!beat}
         This pair of numbers is the standard time-signature of the pattern.
      \item \textbf{Mute Indicator}.
         \index{song editor!mute indicator}
         The letter M is in a black box if the track/pattern is muted, and a
         white box if it is unmuted.
      \item \textbf{Empty}.
         Empty tracks are indicated by a dark-gray filling.
   \end{enumber}

   The patterns column shows a list of all of the patterns that have been
   created in the current song.  Each pattern in this list has a track of
   pattern layouts associated with it in the piano roll section.

   \index{patterns column!left click}
   Left-clicking on the pattern name or the "M" button toggles the muting
   status of the track.

   \index{patterns column!right click}
   Right-clicking on the pattern name or the "M" button brings up the same
   pattern editing menu as discussed in
   \sectionref{subsubsec:seq24_patterns_pattern_filled}.
   Recall that this context menu has the following entries:
   Edit..., Cut, Copy, Song, and Midi Bus.

\subsubsection{Song Editor / Arrangement Panel / Piano Roll}
\label{subsubsec:seq24_song_editor_arrangement_panel_roll}

   The "Piano Roll" section of the arrangement panel is where patterns or
   subsections are inserted, deleted, shrunk, lengthened, or moved.

   Here are the items to note in the Piano Roll area:

   \begin{enumber}
      \item \textbf{Single}.
         In the diagram, under the word "Single", is a very small pattern.
         It is small because it consists only of some MIDI Program Change
         messages meant to set the programs on a Yamaha PSS-790 keyboard.
      \item \textbf{Multiple}.
         This items is the same pattern as in "Single", but dragged out for
         multiple repetitions, simply to show how even the shortest patterns
         can be easily replicated.
      \item \textbf{Pattern Subsection}.
         \index{song editor!middle click}
         \index{pattern subsection}
         Middle-clicking inside a pattern inserts a selection position
         marker in it, breaking the pattern into two equal pieces.
         We call each piece a \textsl{pattern subsection}.
         This division can be done over and over.
         (Note that, in the Song Editor, a middle-click
          \textsl{cannot} be simulated by ctrl-left-click.)
      \item \textbf{Selection Position}.
         A selection position is a marker that divides a pattern into two
         pieces, called \textsl{pattern subsections}.  This makes it easy to
         select smaller portions of a pattern for editing or deleting.  It
         is especially useful for making holes in a pattern.  There may be
         other uses of a selection position that we have not yet discovered.
      \item \textbf{Selection}.
         By clicking inside a pattern or a pattern subsection, it darkens to
         denote that it is selected.
         A pattern subsection can be deleted by the
         \index{keys!delete}
         Delete key, copied by the
         \index{keys!ctrl-c}
         \texttt{Ctrl-C} key, and then inserted (one or more times) by the
         \index{keys!ctrl-v}
         \texttt{Ctrl-V} key.  When inserted, each insert goes immediately
         after the current item or the previous insertion.  The same can be
         done for whole patterns.
      \item \textbf{Section Length}.
         Looking closely at the diagram where the arrows point, small
         squares in the corner of the patterns can be seen.  By grabbing
         that square with a left-click, the square can be moved horizontally
         to either lengthen or shorted the pattern or pattern subsection, if
         there is room to move in the desired direction.
         It doesn't matter if the item is selected or not.
      \item \textbf{Section Movement}.
         If, instead of grabbing the section length handle, one grabs inside
         the pattern or pattern subsection, that item can be moved
         horizontally, as long as their is room.  Or course, left-clicking
         inside the item will also cause it to show as selected.
      \item \textbf{Expansion}.
         Originally, all the long patterns of this song were continuous.
         But, by setting the L and R markers, and using the \textbf{Expand}
         button, we opened up some silent space in the song.
   \end{enumber}

   The \textsl{Sequencer24} help files refer to work in the Song Editor as the
   "Performance Edit" or "Performance Mode".  Adding a pattern in this
   window is a bit like adding a note in the Pattern Editor.
   One clicks, holds, and drags the mouse to insert a copy of the pattern
   associated with the row in which one is dragging.  The longer one drags,
   the more copies of the pattern that are inserted.

   \index{song editor!right click hold}
	Right-click on the arrangement panel (roll) to enter
   draw mode, and hold the button.

   \index{song editor!left click right hold}
   Then left-click the mouse to insert one copy of the pattern.  The
   inserted pattern will show up as a box with a tiny representation of the
   notes visible inside.  (Some patterns, however, can be less than a
   measure in length, resulting in a tiny box.)

   \index{song editor!right left hold drag}
   To keep adding more copies of the pattern, continue to hold both buttons
   and drag the mouse rightward.

   \index{song editor!middle click}
   Middle-click on a pattern to drop a new selection position into the
   pattern,
   \index{song editor!pattern subsection}
   which breaks the pattern into two equal \textsl{pattern subsections}.
   Each middle-click on the pattern adds a new selection position,
   halving the size of the subsections as more pattern subsections are
   added.

   \index{song editor!left click}
   When a pattern or a pattern subsection is left-clicked, it is marked as
   dark gray.
   \index{song editor!right left hold drag}
   When a right-left-hold-drag action is done in this gray area, the result
   is to \textsl{delete} that pattern section or subsection.
   \index{keys!delete}
   One can also hit the Delete key to \textsl{delete} that pattern section
   or subsection.

\subsubsection{Song Editor / Arrangement Panel / Measures Ruler}
\label{subsubsec:seq24_song_editor_arrangement_panel_measures_ruler}

   The \textsl{measures ruler} is the ruled and numbered section at the top
   of the arrangement panel.  It provides a place to put the left and right
   markers.  In the \textsl{Sequencer24} documentation, it is called the "bar
   indicator".

   \index{measures ruler!left-click}
   Left-click in the measures ruler to drop an
   \index{L anchor}
   \textbf{L anchor} on the measures ruler.
   \index{measures ruler!right-click}
   Right-click in the measures ruler to drop an
   \index{R anchor}
   \textbf{R anchor} on the measures ruler.

   Once these anchors are in place, then use
	the \textsl{Collapse} and \textsl{Expand} buttons to modify the
   placement of the pattern events.

   Note that the \textbf{L marker} serves as the start position for playback
   in the Song Editor.  One can change the start position only when the
   performance is not playing.

%-------------------------------------------------------------------------------
% vim: ts=3 sw=3 et ft=tex
%-------------------------------------------------------------------------------
