%-------------------------------------------------------------------------------
% seq24_usr_file
%-------------------------------------------------------------------------------
%
% \file        seq24_usr_file.tex
% \library     Documents
% \author      Chris Ahlstrom
% \date        2015-08-31
% \update      2015-09-01
% \version     $Revision$
% \license     $XPC_GPL_LICENSE$
%
%     Provides the usr_file.
%
%-------------------------------------------------------------------------------

\section{Sequencer24 User Configuration File}
\label{sec:seq24_usr_file}

   \index{sequencer24usr}
   \index{[sequencer24usr]}   % for convenience
   The \textsl{Sequencer24} configuration file was called
   \texttt{.seq24usr}, and it was stored in the user's \texttt{\$HOME}
   directory.
   This is the same name used by \textsl{Seq24}, so we created an new file
   to take its place, with a fall-back to the original file-name if the new
   file does not exist, or if \textsl{Sequencer24} is running in
   legacy mode.

   After you run \textsl{Sequencer24} for the first time, it will generate a
   \texttt{sequencer24usr} file in your home directory:

   \begin{verbatim}
      /home/ahlstrom/.config/sequencer24/sequencer24usr
   \end{verbatim}

   It allows you to give an alias to 
   each MIDI bus, MIDI channel, and MIDI control 
   codes, per channel.
   The name is a bit misleading... do not confuse this file with the
   \texttt{sequencer24rc} file.

   The process for setting up the user file is to:

   \begin{enumber}
      \item Define one or more MIDI busses, the name of each, and what
         instruments are on which channels.  Each buss is configured in a
         section of the form "\textbf{[user-midi-bus-X]}", where "X" ranges
         from 0 on up.
      \item Define all of the instruments and their control-code
         names if they have them.  Each instrument is configured in a
         section of the form "\textbf{[user-instrument-X]}", where "X"
         ranges from 0 on up.
   \end{enumber}

\subsection{Sequencer24 User / MIDI Bus Definitions}
\label{subsec:seq24_usr_file_midi_bus_definitions}

   \index{[user-midi-bus-definitions]}
   This section begins with an
   "INI" group marker \texttt{[user-midi-bus-definitions]}.
   It defines the number of user busses that will be configured in this file.

   \begin{verbatim}
      [user-midi-bus-definitions]
      3     # number of user buses
   \end{verbatim}

   This means that the \texttt{sequencer24usr} file will have three MIDI buss
   sections: [user-midi-bus-0], [user-midi-bus-1], and [user-midi-bus-2].
   Here's is an annoted example of one such section:

   \begin{verbatim}
      [user-midi-bus-0]
      2x2 A (SuperNova,Q,TX81Z,DrumStation)     # name of the device
      16                                        # number of channels

      # NOTE: Channels are 0-15, not 1-16.  Instruments set to -1 = GM

      0 1                                       # channel and instrument
      1 1 
      2 1
      3 1
      4 1
      5 1
      6 1
      7 1
      8 3
      9 3
      10 3
      11 0
      12 0
      13 0
      14 0
      15 2
   \end{verbatim}

   Here's an example of one that needs only one override:

   \begin{verbatim}
      [user-midi-bus-2]
      PCR-30 (303)
      1                                         # number of channels
      0 8                                       # channel and instrument
      # The rest default to -1 - General MIDI
   \end{verbatim}

\subsection{Sequencer24 User / MIDI Instrument Definitions}
\label{subsec:seq24_usr_file_midi_instrument_definitions}

   \index{[user-instrument-definitions]}
   This section begins with an
   "INI" group marker \texttt{[user-instrument-definitions]}.
   It defines the number of user instruments that will be configured in this
   file.

   \begin{verbatim}
      [user-instrument-definitions]
      9     # number of user instrument
   \end{verbatim}

   So this "rc" file will define 9 instruments.  We will provide one section
   as a sample.

   \begin{verbatim}
      [user-instrument-0]
      Waldorf Micro Q                     # name of instrument
      128                                 # number of MIDI controllers
      0                                   # first controller value, unnamed
      1 Modulation Wheel
      2 Breath Control
      3 
      4 Foot Control
      5 Glide Rate
      6 
      7 Channel Volume
      8
      9
      10 Pan
      11 
      12 Arp Range (0-9) (1-10 octaves)
      13 Arp Length (0-15) (1-16 steps)
      14 Arp Active (0-3) (Off,On,One Shot,Hold)
      15 LFO 1 Shape (0-5) (Sine,Tri,Square,Saw,Rand,S&H)
         . . .
      119
      120 All Sound Off (0)
      121 Reset All Controllers (0)
      122 Local Control (0-127) (Off,On)
      123 All Notes Off (0)
      124
      125
      126
      127
   \end{verbatim}

   We assume that an unnamed control number is an unsupported control number.

   Here is an instrument where its synthesis parameters can be controlled:

   \begin{verbatim}
      [user-instrument-1]
      SuperNova
      128
      0 Bank Select MSB
      1 Modulation Wheel
      2 Breath Controller
      3 Arp Pattern Select
      4 Ring Modulator 2 * 3 Mix Level
      5 Portamento Time
      6 Data Entry
      7 Part / Program Volume
      8 Effects Confg Morph Amount
      9 Arp Speed (Internal Clock Rate) [*]
      10 Pan
      11 Osc 1 Fine Tune
      12 Osc 3 Fine Tune
      13 Osc 1 Soften
      14 Osc 2 Soften
      15 Osc 3 Soften
      16 LFO 1 Speed
      17 LFO 1 Delay
         . . .
      119 Delay Mod Wheel Depth
      120 All Sound Off
      121 Reset Controllers
      122 Local Control [*]
      123 All Notes Off
      124 All Notes Off
      125 All Notes Off
      126 All Notes Off
      127 All Notes Off
   \end{verbatim}

   Here is an instrument that perhaps has no controllers, or maybe is simply
   not configured yet.

   \begin{verbatim}
      [user-instrument-4]
      WaveStation
      0
   \end{verbatim}

   The sample file \texttt{contrib/scripts/dot-seq24usr} contains examples
   of some other kinds of instruments, such as drum machines.

   Sometime we would like to create a section that sets up the
   \textsl{Yoshimi} 1.3.5+ software synthesizer as an instrument.

%-------------------------------------------------------------------------------
% vim: ts=3 sw=3 et ft=tex
%-------------------------------------------------------------------------------
