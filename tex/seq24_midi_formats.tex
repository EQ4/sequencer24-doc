%-------------------------------------------------------------------------------
% seq24_midi_formats
%-------------------------------------------------------------------------------
%
% \file        seq24_midi_formats.tex
% \library     Documents
% \author      Chris Ahlstrom
% \date        2015-09-03
% \update      2015-09-04
% \version     $Revision$
% \license     $XPC_GPL_LICENSE$
%
%     Provides a discussion of the formats (legacy and new) of the last
%     track of an Seq24/Sequencer24 MIDI file.
%
%-------------------------------------------------------------------------------

\section{Proprietary Track Format and Other MIDI Notes}
\label{sec:proprietary_track_and_midi_notes}

\subsection{Legacy Proprietary Track Format}
\label{subsec:legacy_proprietary_track_format}

   Before we get to the last, proprietary track, note that the tracks that
   precede it include the SeqSpec ("sequencer-specific", sort of)
   control tags shown in table~\ref{table:seqspec_items_normal_tracks}.

   \begin{table}[htb]
      \centering
      \caption{SeqSpec Items in Normal Tracks}
      \label{table:seqspec_items_normal_tracks}
      \begin{tabular}{l l}
         \texttt{c\_midibus}        & \texttt{24 24 00 01 00 00 00 00} \\
         \texttt{c\_midich}         & \texttt{24 24 00 02 00 00 00 00} \\
         \texttt{c\_timesig}        & \texttt{24 24 00 06 00 00 00 00} \\
         \texttt{c\_triggers\_new}  & \texttt{24 24 00 08 00 00 00 00} \\
      \end{tabular}
   \end{table}

   Note that these tags (not created by the application, and not present in
   the proprietary track) are preceded by the standard MIDI
   "FF 75 len" meta-event sequence.

   After all the counted MIDI tracks are read, \textsl{Seq24} checks for
   extra data.  If there is extra data, \textsl{Seq24} reads a long value.
   The first one encountered is a MIDI "sequencer-specific"
   (\textsl{SeqSpec}) section.  It starts with

   \begin{verbatim}
      0x24240010 == a Seq24 c\_midictrl proprietary value
   \end{verbatim}

   Getting this value first is simplified MIDI in two ways.
   First, the second does not begin with any kind of track marker.  MIDI
   requires an "MTrk" marker to start a track, though it also requires
   unknown markers to be supported.  Some applications, like
   \textsl{timidity}, handle this situation.  Others, like \textsl{midicvt},
   complain about an unexpected header marker.
   Second, normally, MIDI wants to see the triad of

   \begin{verbatim}
      status = FF, type= 7F (proprietary), length = whatever
   \end{verbatim}

   to precede proprietary data.
   Now, as the table below (REFERENCE) shows, most applications accept
   the shortcut legacy format, but \textsl{midicvt} does not.

   So, as a "bug" fix, we want to be able to write and read
   this information properly in \textsl{Sequencer24}.
   We also need to be able to read legacy Seq24 MIDI files.

   Any way, we have the \textbf{c\_midictrl} information now.  Next, we read
   a long value, seqs.  It is 0.

   \begin{verbatim}
      24 24 00 10 00 00 00 00
   \end{verbatim}

   Read the next long value, 0x24240003.  This is \textbf{c\_midiclocks}.
   We get a value of 0 for "TrackLength" (now a local variable called
   "busscount"):

   \begin{verbatim}
      24 24 00 03 00 00 00 00
   \end{verbatim}

   If busscount were greater than 0, then for each value, we would read a
   byte value represent the bus a clock was on, and setting the clock value
   of the master MIDI buss.

   Another check for more data is made.

   \begin{verbatim}
      24 24 00 05 00 20 00 00
   \end{verbatim}

   0x24240005 is \textbf{c\_notes}.  The value screen\_sets is read (two
   bytes) and
   here is 0x20 = 32.  For each screen-set:

   \begin{verbatim}
      len = read\_short()
   \end{verbatim}

   If non-zero, each of the \texttt{len} bytes is appended as a string.
   Here, len is 0 for all 32 screensets, so the screen-set notepad is set to
   an empty string.

   Another check for more data is made.

   \begin{verbatim}
      24 24 00 07 00 00 00 78
   \end{verbatim}

   0x24240007 is \textbf{c\_bpmtag}.  The long value is read and sets the
   perform object's bpm value.  Here, it is 120 bpm.

   Another check for more data is made.

   \begin{verbatim}
      24 24 00 09 00 00 04 00
   \end{verbatim}

   0x24240009 is \textbf{c\_mutegroups}.  The long value obtained here is
   1024.  If this value is not equal to the constant
   \textbf{c\_gmute\_tracks} (1024), a warning is emitted to the console,
   but processing continues anyway, 32 x 32 long values are read to select
   the given group-mute, and then set each of its 32 group-mute-states.

   In our sample file, 32 groups are specified, but all 32 group-mute-state
   values for each are 0.

   So, to summarize the legacy proprietary track's data, ignoring the data
   itself, which is mostly 0 values, as shown in
   table~\ref{table:seqspec_items_legacy_track}

   \begin{table}[htb]
      \centering
      \caption{SeqSpec Items in Legacy Proprietary Track}
      \label{table:seqspec_items_legacy_track}
      \begin{tabular}{l l}
\texttt{c\_midictrl}    & \texttt{24 24 00 10 00 00 00 00} \\
\texttt{c\_midiclocks}  & \texttt{24 24 00 03 00 00 00 00} (buss count = 0) \\
\texttt{c\_notes}       & \texttt{24 24 00 05 00 20 00 00} (screen sets = 32) \\
\texttt{c\_bpmtag}      & \texttt{24 24 00 07 00 00 00 78} (bpm = 120) \\
\texttt{c\_mutegroups}  & \texttt{24 24 00 09 00 00 04 00} (mg = 1024) \\
      \end{tabular}
   \end{table}

   The new format (again, ignoring the data) takes up a few more bytes.
   It starts with the normal track marker and size data, followed by a
   made-up track name ("Sequencer24-S"),
   as shown in table~\ref{table:seqspec_items_new_track}.

   \begin{table}[htb]
      \centering
      \caption{SeqSpec Items in New Proprietary Track}
      \label{table:seqspec_items_new_track}
      \begin{tabular}{l l}
\texttt{"MTrk" etc.}   & \texttt{4d 54 72 6b 00 00 11 0d 00 ...} \\
\texttt{Track name}    & \texttt{53 65 71 75 65 6e 63 65 72 32 34 2d 53} \\
\texttt{c\_midictrl}   & \texttt{ff 7f 04 24 24 00 10 00} (???) \\
\texttt{c\_midiclocks} & \texttt{ff 7f 04 24 24 00 03 00} (buss count = 0) \\
\texttt{c\_notes}      & \texttt{ff 7f 46 24 24 00 05 00 20 00...} (screen sets = 32) \\
\texttt{c\_bpmtag}     & \texttt{ff 7f 08 24 24 00 07 00 00 00 78} (bpm = 120) \\
\texttt{c\_mutegroups} & \texttt{ff 7f a1 08 24 24 00 09 00 00 04 00...} (mg = 1024) \\
      \end{tabular}
   \end{table}

For the new format, the components of the final proprietary track size are
as shown here:

   \begin{enumber}
      \item \textbf{Delta time}.  1 byte, always 0x00.
      \item \textbf{Sequence number}.  5 bytes.  OPTIONAL.
      \item \textbf{Track name}. 3 + 10 or 3 + 15
      \item \textbf{Series of proprietary specs}:
      \begin{itemize}
         \item \textbf{Prop header}:
         \begin{itemize}
            \item If legacy format, 4 bytes.
            \item Otherwise, 2 bytes + varinum\_size(length) + 4 bytes.
            \item Length of the prop data.
         \end{itemize}
      \end{itemize}
      \item \textbf{Track End}. 3 bytes.
   \end{enumber}

\subsection{MIDI Information}
\label{subsec:midi_information}

   This section just provides some useful, basic information about MIDI
   data.

\subsubsection{MIDI Variable-Length Value}
\label{subsec:midi_variable_length_value}

   Length of a variable length value:

   https://en.wikipedia.org/wiki/Variable-length\_quantity

   \begin{verbatim}
      1 byte:  0x00 to 0x7F
      2 bytes: 0x80 to 0x3FFF
      3 bytes: 0x4000 to 0x001FFFFF
      4 bytes: 0x200000 to 0x0FFFFFFF
   \end{verbatim}

\subsubsection{MIDI Track Chunk}
\label{subsec:midi_track_chunk}

   Track chunk == MTrk + length + track\_event [+ track\_event ...]

   \begin{itemize}
      \item \textsl{MTrk} is 4 bytes representing the literal string "MTrk".
         This marks the beginning of a track.
      \item \textsl{length} is 4 bytes the number of bytes in the track
         chunk following this number.  That is, the marker and length are
         not counted in the length value.
      \item \textsl{track\_event} denotes a sequenced track event; usually
         there are many track events in a  track.  However, some of the
         events may simply be informational, and not modify the audio
         output.
   \end{itemize}

   A track event consists of a delta-time since the last event, and one of
   three types of events.
 
   track\_event = v\_time + midi\_event | meta\_event | sysex\_event
 
   \begin{itemize}
      \item \textsl{v\_time} is the variable length value for elapsed time
         (delta time) from the previous event to this event.
      \item \textsl{midi\_event} is any MIDI channel message such as note-on
         or note-off.
      \item \textsl{meta\_event} is an SMF meta event.
      \item \textsl{sysex\_event} is an SMF system exclusive event.
   \end{itemize}

\subsubsection{MIDI Meta Events}
\label{subsec:midi_meta_events}

   Meta events are non-MIDI data of various sorts consisting of a fixed prefix,
   type indicator, a length field, and actual event data..
 
   meta\_event = 0xFF + meta\_type + v\_length + event\_data\_bytes

   \begin{itemize}
      \item \textsl{meta\_type} is 1 byte, expressing one of the meta event
         types shown in the table that follows this list.
      \item \textsl{v\_length} is length of meta event data, a variable
         length value.
      \item \textsl{event\_data\_bytes} is the actual event data.
   \end{itemize}

   \begin{table}
      \centering
      \caption{MIDI Meta Event Types}
      \label{table:midi_meta_event_types}
      \begin{tabular}{l l}
         Type	& Event \\
         0x00	& Sequence number \\
         0x01	& Text event \\
         0x02	& Copyright notice \\
         0x03	& Sequence or track name \\
         0x04	& Instrument name \\
         0x05	& Lyric text \\
         0x06	& Marker text \\
         0x07	& Cue point \\
         0x20	& MIDI channel prefix assignment \\
         0x2F	& End of track \\
         0x51	& Tempo setting \\
         0x54	& SMPTE offset \\
         0x58	& Time signature \\
         0x59	& Key signature \\
         0x7F	& Sequencer specific event \\
      \end{tabular}
   \end{table}

   \textsl{Timidity} reads the legacy and new formats and plays the tune.
   It saves the "b4uacuse" tune out, in both formats, with a "MIDI
   divisions" value of 192, versus its original value of 120.  The song
   plays a little bit faster after this conversion.

   The \textsl{midicvt} application does not read the legacy format.  It
   expects to see the MTrk marker.  Even if the --ignore option is provided,
   \textsl{midicvt} does not like the legacy \textsl{Seq24} format.
   However, as table~\ref{table:midi_file_support_table}
   shows, most applications are more
   forgiving, and can read (or ignore) the legacy format.  The
   \textsl{gsequencer} has some major issues, but it is probably our setup
   (no JACK running?)

   \begin{table}
      \centering
      \caption{Application Support for MIDI Files}
      \label{table:midi_file_support_table}
      \begin{tabular}{l l l l}
         \textbf{Application}  &
            \textbf{Legacy} &
            \textbf{New} & 
            \textbf{Original File} \\
         ardour       & TBD       & TBD       & TBD \\
         composite    & TBD       & TBD       & TBD \\
         gsequencer   & No        & No        & No \\
         lmms         & Yes       & Yes       & Yes \\
         midi2ly      & Yes       & Yes       & TBD \\
         midicvt      & No        & Yes       & Yes \\
         midish       & TBD       & TBD       & TBD \\
         muse         & TBD       & TBD       & TBD \\
         playmidi     & TBD       & TBD       & TBD \\
         pmidi        & TBD       & TBD       & TBD \\
         qtractor     & Yes       & Yes       & Yes \\
         rosegarden   & Yes       & Yes       & Yes \\
         superlooper  & TBD       & TBD       & TBD \\
         timidity     & Yes       & Yes       & Yes \\
      \end{tabular}
   \end{table}

%-------------------------------------------------------------------------------
% vim: ts=3 sw=3 et ft=tex
%-------------------------------------------------------------------------------
