%-------------------------------------------------------------------------------
% seq24-user-manual
%-------------------------------------------------------------------------------
%
% \file        seq24-user-manual.tex
% \library     Documents
% \author      Chris Ahlstrom
% \date        2015-07-19
% \update      2015-08-30
% \version     $Revision$
% \license     $XPC_GPL_LICENSE$
%
%     This document provides LaTeX documentation for seq24.
%
%-------------------------------------------------------------------------------

\documentclass[
 11pt,
 twoside,
 a4paper,
 headinclude,
 footinclude,
 final                                 % versus draft
]{article}

\input{yoshimi-docs-structure}         % specifies document structure and layout

\makeindex

\begin{document}

\title{A Seq24 User Manual}
\author{Chris Ahlstrom\\
   (\texttt{ahlstromcj@gmail.com})}
\date{\today}
\maketitle
\tableofcontents
\listoffigures                         % print the list of figures
% \listoftables                        % print the list of tables

% Change the paragraph style to remove indenting and put a line between each
% paragraph.  This could be moved up into the preamble, but then would
% affect the spacing of the TOC and LOF, LOT noted above.

\setlength{\parindent}{0pt}
\setlength{\parskip}{1ex plus 0.5ex minus 0.2ex}

\section{Introduction}
\label{sec:introduction}

   This document describes how to use \textsl{Seq24} \cite{seq24},
   through version 0.9.2, dated approximately to 2010.

   \textsl{Seq24} is
   a live-looping sequencer with an interface more like a hardware sequencer
   than the typical software MIDI sequencer.  The contents of this document
   are derived in part from existing documents, such as the \texttt{SEQ24}
   file shipped with the \textsl{Seq24} source code, and a printout from a
   long-lost wiki.

   \textsl{Seq24} is not a synthesizer.  One needs hardware synthesizer, or
   a software synthesizer such as
   Timidity \cite{timidity},
   FluidSynth \cite{fluidsynth},
   ZynAddSubYX \cite{zynaddsubfx} and Yoshimi \cite{yoshimi} \cite{yoshimi2},
   AmSynth \cite{amsynth},
   Bristol \cite{bristol},
   and others (see \cite{linuxsynths} for a fairly comprehensive list of
   "Linux" synthesizers).

   \textsl{Seq24} is meant to work a bit like an Alesis SR16 drum machine,
   which, for some, is a very intuitive and fast way to do MIDI.
   If one has worked with trackers like \textsl{SoundTracker} and
   \textsl{ShakeTracker}, then "you are a tracker guy and it gonna go fast".
   With \textsl{Seq24}, one creates several patterns, and then combines them.

   There are a number of current authors of \texttt{Seq24} today,
   as one can see in
   \figureref{fig:seq24_menu_help_credits},
   and it
   \figureref{fig:seq24_menu_help_doc}.
   The original author is Rob C. Buse; where the word "I" occurs, that is
   probably him.

   Now, unfortunately, \textsl{Seq24} is not currently under active
   development (2010 seems to be the last update).  There are a number of
   half-hearted forks for it on \textsl{GitHub}, including some conversions
   to Java-based and browser-based code.  One of the fork consists of some
   patches for LASH support, which should be incorporated at some point.
   \textsl{Seq24} could use a few more features, such as "infinite
   patterns" (i.e. tracks), better annunciation of mouse modes,
   ways to work around the need for two mice, a more up-to-date GUI,
   a few bug fixes, fix the formating of out rc files,
   and a way to edit parts of the MIDI file textually.

   So, why would we bother with \textsl{Seq24}?  The next section answers,
   in pretty much the original author's words.

\subsection{Seq24: What and Why?}
\label{subsec:introduction_seq24_vs_others}

   \texttt{Seq24} is a real-time MIDI sequencer. It was created to 
   provide a very simple interface for editing and playing 
   MIDI 'loops'. After searching for a software based 
   sequencer that would provide the functionality needed for 
   a live performance, there was little found in the 
   software realm. I set out to create a very minimal sequencer 
   that excludes the bloated features of the large software 
   sequencers, and includes a small subset of features that 
   I have found usable in performing. 

   \begin{quotation}
      Written by Rob C. Buse.  I wrote this program to fill a
      hole.  I figure it would be a waste if I was the only one
      using it.  So, I released it under the GPL.
   \end{quotation}

   Taking advantage of Rob's migh morality, we've creating a continuation,
   minor fork, refactoring, fixing, improvements (we hope) of \textsl{Seq24}
   in a project called "Sequencer24" \cite{sequencer24}.  It preserves the
   lean nature of \textsl{Seq24}, and might even make it a bit leaner, while
   adding a few features we've found useful.  Stay tuned for a project that
   documents "Sequencer24" and its new features.

\subsection{Document Structure}
\label{subsec:introduction_document_structure}

   The structure of this document is based on the user-interface of
   \textsl{Seq24}.
   The sections are basically provided
   in the order their contents appear in the user interface of
   \textsl{Seq24}.  To help the reader jump around this document, multiple
   links and references are supplied.

   Usage tips
   \index{tips!in document}
   for each of the functions provided in
   \textsl{Seq24} are sprinkled throughout this document.
   Each tip occurs in a section beginning with "Tip:".
   Each tip is provided with an entry in the Index, under the
   main topic "tips".

   Bug notes
   \index{bugs!in document}
   for some of the oddities found in \textsl{Seq24} are
   sprinkled throughout this document.
   Each bug occurs in a sentence beginning with "Bug:".
   Each bug is provided with an entry in the Index, under the
   main topic "bugs".

   TODO items
   \index{todo!in document}
   are also present, in the same vein.
   This document currently has a lot of them!

\subsection{Let's Get Started!}
\label{subsec:introduction_lets_get_started}

   Let us run \textsl{Seq24}, but run it without using \textsl{JACK}, which
   complicates the discussion of \textsl{Seq24}.  The first
   thing to do is make sure one has no other sound application running
   (unless one wants to risk blocking \textsl{Seq24} or hearing two sounds
   simultaneously, depending on one's sound card and ALSA setup).
   Then start \textsl{Seq24} so that it uses ALSA for
   MIDI.  Provide a default MIDI file so that all elements of the user
   interface can come into play.
   Also use the "\&" character so that we get back to the
   command-line prompt.
%   See \sectionref{sec:seq24_man_page}.

\begin{verbatim}
   $ seq24 click_4_4.midi &
\end{verbatim}

\begin{figure}[H]
   \centering 
   \includegraphics[scale=0.75]{seq24-first-screen.png}
   \caption{Seq24 Main Screen}
   \label{fig:seq24_main_screen}
\end{figure}

%  \includegraphics[scale=1.0]{seq24-first-screen.jpg}
%  That figure a bit out-of-date (it does not show the new "Part of"
%  feature), so we've replaced it with a new screen capture.
   
   Then the \textsl{Seq24} main window appears, as shown in
   \figureref{fig:seq24_main_screen}.

   \index{tooltips}
   As with most user-interfaces, holding the mouse over any button for a
   short period will let one view a short description (tooltip)
   of what it does.

   The \textsl{Seq24} program is basically a loop playing machine with a 
   simple interface.  Before we describe this interface, it is useful to
   present some concepts.

% Important Concepts

%-------------------------------------------------------------------------------
% seq24_concepts
%-------------------------------------------------------------------------------
%
% \file        seq24_concepts.tex
% \library     Documents
% \author      Chris Ahlstrom
% \date        2015-07-19
% \update      2015-08-30
% \version     $Revision$
% \license     $XPC_GPL_LICENSE$
%
%     Provides the concepts.
%
%-------------------------------------------------------------------------------

\section{Concepts}
\label{sec:concepts}

   This section presents some useful concepts and definitions of terms as
   they are used in \textsl{Seq24}.

\subsection{Concepts / Terms}
\label{subsec:concepts_terms}

   This section doesn't provide comprehensive coverage of terms.  It
   covers mainly terms that puzzled the author at first or that are
   necessary to understand the \textsl{Seq24} program.

\subsubsection{Concepts / Terms / armed}
\label{subsubsec:concepts_terms_armed}

   \index{armed}
   An armed sequence is a sequence that will be heard.
   "Armed" is the opposite of "muted".
   Performing an \textsl{arm} operation in \textsl{Seq24} means clicking on
   an "unarmed" sequence in the patterns panel (the main window of
   \textsl{Seq24}).  An unarmed sequence will not be heard, and it
   has a white background.  When the sequence is \textsl{armed},
   it will be heard, and it has a black background.

   A sequence can be armed or unarmed in three ways:

   \begin{itemize}
      \item Clicking on the sequence/pattern box.
      \item Pressing the hot-key for that sequence/pattern box.
      \item Opening up the Song Editor and starting playback; the
            sequences arm/unarm depending on the layout of the
            sequences in the piano roll of the Song Editor.
   \end{itemize}

\subsubsection{Concepts / Terms / buss (bus)}
\label{subsubsec:concepts_terms_buss}

   \index{bus}
   \index{buss}
   A \textsl{buss} (also spelled "bus" these days) is an entity onto which
   MIDI events can be placed, in order to be heard or to affect the
   playback.

\subsubsection{Concepts / Terms / group}
\label{subsubsec:concepts_terms_group}

   \index{group}
   A \textsl{group} in \textsl{Seq24} is one of up to 32 previously-defined
   mute/unmute patterns in the active screen set.
   A group is a set of patterns that can toggle their playing state
   together.  Every group contains all 32 sequences in the active screen
   set.  This concept is similar to mute/unmute groups in hardware
   sequencers.

\subsubsection{Concepts / Terms / loop}
\label{subsubsec:concepts_terms_loop}

   \index{loop}
   \textsl{Loop}
   is a synonym for \textsl{pattern} or \textsl{sequence}, when used
   in existing \textsl{Seq24} documentation.
   Each loop is represented by a box in the Patterns window.

\subsubsection{Concepts / Terms / measures ruler}
\label{subsubsec:concepts_terms_measures_ruler}

   \index{measures ruler}
   The \textsl{measures ruler} is the bar at the top of the Song Editor
   arrangement window that shows the numbering of the measures in the song.
   Left and right markers can be dropped on this ruler to set durations to
   be played, looped, expanded, or collapsed.

   Note:
   \index{bar indicator}
   The original \textsl{Seq24} documentation calls this item the
   \textsl{bar indicator}.

\subsubsection{Concepts / Terms / MIDI clock}
\label{subsubsec:concepts_terms_midi_clock}

   \textsl{MIDI clock} is
   \index{midi clock}
   a MIDI timing reference signal used to synchronize pieces of equipment
   together. MIDI clock runs at a rate of 24 ppqn (pulses per quarter note).
   This means that the actual speed of the MIDI clock varies with the tempo
   of the clock generator (as contrasted with time code, which runs at a
   constant rate).

\subsubsection{Concepts / Terms / pattern}
\label{subsubsec:concepts_terms_pattern}

   A \textsl{Seq24} \textsl{pattern}
   \index{pattern}
   (also called a "sequence" or "loop")
   is a short unit of melody or rhythm in \textsl{Seq24},
   extending for a small number of measures (in most cases).
   Each pattern is represented by a box in the Patterns window.

   Each pattern is editable on its own.  All patterns can be layed out in
   a particular arrangement to generate a more complex song.

   \textsl{pattern} is a synonym for \textsl{loop} or \textsl{sequence}.
   It is our preferred term.

\subsubsection{Concepts / Terms / performance}
\label{subsubsec:concepts_terms_performance}

   In the jargon of \textsl{Seq24}, a
   \index{performance}
   \textsl{performance} is an organized collection of patterns.
   This collection of patterns is created using the Song Editor.

\subsubsection{Concepts / Terms / queue mode}
\label{subsubsec:concepts_terms_queue_mode}

   TODO:  \index{todo!queue definition}
   We need to figure out what "queue" refers to in the documentation, what
   do the "temporary queue" and "permanent queue" are, and what the "keep
   queue" functionality is used for.

\subsubsection{Concepts / Terms / screen set}
\label{subsubsec:concepts_terms_screen_set}

   The \textsl{screen set}
   \index{screen set}
   is a set of patterns that fit within the 8x4 grid of loops/pattern in the
   Patterns panel.
   \textsl{Seq24} supports multiple screens sets, and a name can be given to
   each for clarity.

\subsubsection{Concepts / Terms / sequence}
\label{subsubsec:concepts_terms_sequence}

   \index{sequence}
   \textsl{Sequence} seems to be
   another synonym for \textsl{pattern}, used in some of the \textsl{Seq24}
   documentation.  \textsl{Loop} is another synonym.
   Each sequence is represented by a box in the Patterns window.

\subsubsection{Concepts / Terms / snapshot}
\label{subsubsec:concepts_terms_snapshot}

   \index{snapshot}
   A \textsl{Seq24} \textsl{snapshot} is simply a briefly preserved state.
   One can press a snapshot key, change the state of the patterns for live
   playback, and then release the snapshot key to revert to the state when
   it was first pressed.  (Might call it a "revert" key, instead.)

\subsubsection{Concepts / Terms / song}
\label{subsubsec:concepts_terms_song}

   \index{song}
   A \textsl{song} is a collection of patterns in a specific layout, as
   assembled via the Song Editor window.

%-------------------------------------------------------------------------------
% vim: ts=3 sw=3 et ft=tex
%-------------------------------------------------------------------------------


% Menu

%-------------------------------------------------------------------------------
% seq24_menu
%-------------------------------------------------------------------------------
%
% \file        seq24_menu.tex
% \library     Documents
% \author      Chris Ahlstrom
% \date        2015-08-31
% \update      2015-09-01
% \version     $Revision$
% \license     $XPC_GPL_LICENSE$
%
%     Provides the Menu section of seq24-user-manual.tex.
%
%-------------------------------------------------------------------------------

\section{Menu}
\label{sec:seq24_menu}

   The \textsl{Sequencer24} menu, as seen at the top of
   \figureref{fig:seq24_main_screen},
   is fairly simple, but it is important to understand the
   structure of the menu entries.

\subsection{Menu / File}
\label{subsec:seq24_menu_file}

   The \textbf{File} menu is used to save and load standard 
   MIDI files.  It should be able to handle any 
   Format 1 standard files that any other sequencer 
   is capable of exporting.  

   The \textsl{Sequencer24}
   menu entry contains the sub-items shown in
   \figureref{fig:seq24_menu_file_items}.
   The next few sub-sections discuss the sub-items in the 
   \textsl{File} sub-menu.

\begin{figure}[H]
   \centering 
   \includegraphics[scale=0.75]{menu/menu_file.png}
   \caption{Sequencer24 File Menu Items}
   \label{fig:seq24_menu_file_items}
\end{figure}

   \begin{enumber}
      \item \textbf{New}
      \item \textbf{Open...}
      \item \textbf{Save}
      \item \textbf{Save As...}
      \item \textbf{Import...}
      \item \textbf{Options...}
      \item \textbf{Exit}
   \end{enumber}

\subsection{Menu / File / New}
\label{subsec:menu_file_new}

   The \textbf{New} menu entry clears out any current song and patterns,
   allowing one to create news ones from scratch.
   If unsaved changes are pending, the user will be prompted to save the
   changes.
   Currently, the detection of situations requiring a save (or not requiring
   a save) needs a bit of work.

\subsubsection{Menu / File / Open}
\label{subsubsec:seq24_menu_file_open}

   The \textbf{Open} menu entry opens a song that had been saved previously.
   It opens up a standard GTK+ file dialog.

\begin{figure}[H]
   \centering 
   \includegraphics[scale=0.65]{menu/menu_file_open.png}
   \caption{File Open}
   \label{fig:seq24_menu_file_open}
\end{figure}

   If unsaved changes are pending, the user will be prompted to save the
   changes.

\subsubsection{Menu / File / Save and Save As}
\label{subsubsec:menu_file_open_save_as}

   The \textbf{Save} menu entry saves the song under its current name.
   If there is no current name, then
   it opens up a standard GTK+ file dialog.

   The \textbf{Save As} menu entry saves a song under a different name.
   It opens up the following standard GTK+ file dialog.

\begin{figure}[H]
   \centering 
   \includegraphics[scale=0.65]{menu/menu_file_save_as.png}
   \caption{File Save As}
   \label{fig:seq24_menu_file_save_as}
\end{figure}

   To save a new file, or to save the current existing file to a new name,
   enter the name in the name field, \textsl{without an extension}.
   \textsl{Sequencer24} will append a \texttt{.midi} extension to the filename.

   The file will be save in a format the the Linux \textsl{file} command
   will tag as something like:

   \begin{verbatim}
      myfile.midi: Standard MIDI data (format 1) using 16 tracks at 1/192
   \end{verbatim}

   \index{todo!solve seq24 format}
   It looks like a simple MIDI file, and yet, if one re-opens it in
   \textsl{Sequencer24}, one sees that all of the labelling, pattern information,
   and song layout has been preserved in this file.
   Even the pattern subsections, as discussed in
   \sectionref{subsubsec:seq24_song_editor_arrangement_panel_roll},
   have been saved.
   (But the L and R marker positions are not saved.)

   Compare the sizes of the original project MIDI file,
   \texttt{contrib/b4uacuse.mid}, and the output MIDI file after
   \textsl{Sequencer24} saved the patterns and the song layout we created,
   \texttt{contrib/b4uacuse-seq24.midi}.  The latter is a lot
   bigger.  

\subsubsection{Menu / File / Import}
\label{subsubsec:seq24_menu_file_import}

   The \textbf{Import} menu entry allows one to import a MIDI file
   into a pattern.

\begin{figure}[H]
   \centering 
   \includegraphics[scale=0.65]{menu/menu_file_import.png}
   \caption{File Import}
   \label{fig:seq24_menu_file_import}
\end{figure}

   When imported, each track, whether a music track or an information track,
   is entered into its own loop/pattern box.  The import operation can
   handle reasonably complex files, as shown in the following diagram, which
   shows an import of the \texttt{contrib/b4uacuse.mid} file, which contains
   a transcription of an Eric Clapton tune that we'd made over 20 
   years ago and had uploaded to the \textsl{GEnie} network service.

\begin{figure}[H]
   \centering 
   \includegraphics[scale=0.90]{menu/imported_midi_song.png}
   \caption{Imported MIDI Song}
   \label{fig:seq24_imported_midi_song}
\end{figure}

   Unfortunately, this song was created before the days of General MIDI.
   It is scored for the Yamaha PSS-790 consumer-level synthesizer.
   One can use our MIDI-conversion project (see reference \cite{midicvt}) 
   to convert it to General MIDI format, including General MIDI drums.

\subsubsection{Menu / File / Options}
\label{subsubsec:seq24_menu_file_options}

   The \textbf{Options} menu item provides a number of settings in one
   tabbed dialog, shown in the figure below.
   This dialog allows one to select which sequence gets the MIDI
   clock, which incoming MIDI events control the sequencer, what keys are
   mapped to functions, how the mouse works, and some JACK parameters.

\paragraph{Menu / File / Options / MIDI Clock}
\label{paragraph:seq24_menu_file_options_midi_clock}

   The \textbf{MIDI Clock} tab provides a way to send the MIDI clock to one
   or more of the \textsl{Sequencer24} output busses.
   It is used to configure to what busses the MIDI clock gets dumped.

\begin{figure}[H]
   \centering 
   \includegraphics[scale=0.75]{menu/menu_file_options_midi_clock.png}
   \caption{File / Options / MIDI Clock}
   \label{fig:seq24_menu_file_options_midi_clock}
\end{figure}

   The following elements are present in this dialog:

   \begin{enumber}
      \item \textbf{Buss Name}
      \item \textbf{Off}
      \item \textbf{On (Pos)}
      \item \textbf{On (Mod)}
      \item \textbf{Clock Start Modulo}
   \end{enumber}

   \setcounter{ItemCounter}{0}      % Reset the ItemCounter for this list.

   \itempar{Buss Name}{midi clock!buss name}
   These labels indicate the output busses of \textsl{Sequencer24}.
   They range from \textbf{[1] seq24 1}
   to \textbf{[16] seq24 16}.

   \itempar{Off}{midi clock!off}
   This setting disables the MIDI clock for the given output buss.

   \itempar{On (Pos)}{midi clock!on (pos)}
   The MIDI clock will be sent to this buss.
   MIDI Song Position and MIDI Continue will be sent if playback is starting
   at greater than tick 0 in Song mode.  Otherwise, MIDI Start will be sent.

   \itempar{On (Mod)}{midi clock!on (mod)}
   The MIDI clock will be sent to this buss.
   MIDI Start will be sent and clocking will begin
   once the Song Position has reached the start modulo of the specified size
   (see the next item's description).
   This setting is used for gear that does not respond to Song Position.

   \itempar{Clock Start Modulo}{midi clock!clock start modulo}
   Clock Start Modulo (1/16 Notes).
   This value starts at 1 and ranges on upward to 16384.
   It  defaults to 64.
   It is used by the \textbf{On (Mod)} setting discussed above.
   It is the \texttt{[midi-clock-mod-ticks]} option in the \textsl{Sequencer24}
   "rc" file as described in
   \sectionref{subsec:seq24_rc_file_other_midi}.


\paragraph{Menu / File / Options / MIDI Input}
\label{paragraph:seq24_menu_file_options_midi_input}

   The only item in the \textbf{MIDI Input} tab is the single MIDI input
   buss provided by \textsl{Sequencer24}:  \textbf{[0] seq24 0}.

\begin{figure}[H]
   \centering 
   \includegraphics[scale=0.75]{menu/menu_file_options_midi_input_condensed.png}
   \caption{File / Options / MIDI Input (Condensed View)}
   \label{fig:seq24_menu_file_options_midi_input}
\end{figure}

   This item, if checked allows \textsl{Sequencer24} to be used to record MIDI
   information from another source, or pass it through to the output busses
   that are configured
   to allow pass-through
   (in the Pattern Editor, as discussed in 
   \sectionref{subsec:seq24_pattern_editor_bottom}.)

\paragraph{Menu / File / Options / Keyboard }
\label{paragraph:seq24_menu_file_options_keyboard}

   \textsl{Sequencer24}, as befits a good application, allows extensive use of
   keyboard shortcuts to make operations go faster than when using a mouse.
   The \textbf{Keyboard} tab allows for the configuration of these keyboard
   shortcuts.

\begin{figure}[H]
   \centering 
   \includegraphics[scale=0.75]{menu/menu_file_options_keyboard.png}
   \caption{File / Options / Keyboard}
   \label{fig:seq24_menu_file_options_keyboard}
\end{figure}

   We won't attempt to cover every user-interface item in this busy
   dialog, just the categories.

   \begin{enumber}
      \item \textbf{Show key labels on sequences}
      \item \textbf{Control keys}
      \item \textbf{Sequence toggle keys}
      \item \textbf{Mute-group slots}
      \item \textbf{Learn}
      \item \textbf{Disable}
      \item \textbf{Enable}
   \end{enumber}

   \setcounter{ItemCounter}{0}      % Reset the ItemCounter for this list.

   \itempar{Show key labels on sequence}{keyboard!show labels}
   This item, if enabled, shows the key labels in the lower-right corner of
   each loop/pattern in the Patterns window.

   \itempar{Control keys}{keyboard!control keys}
   This block of fields provides shortcut keys for many operations of
   \textsl{Sequencer24}.

   \begin{enumber}
      \item \textbf{Start}.
         Key: \index{keys!space} \textbf{space}.
      \item \textbf{Stop}.
         Key: \index{keys!esc} \textbf{Escape}.
      \item \textbf{Snapshot 1}.
         Key: \index{keys!alt-l} \textbf{Alt\_L}.
      \item \textbf{Snapshot 2}.
         Key: \index{keys!alt-r} \textbf{Alt\_R}.
      \item \textbf{bpm up}.
         Key: \index{keys!apostrophe} \textbf{apostrophe}.
      \item \textbf{bpm down}.
         Key: \index{keys!semicolon} \textbf{semicolon}.
      \item \textbf{Replace}.
         Key: \index{keys!ctrl-l} \textbf{Control\_L}.
      \item \textbf{Queue}.
         Key: \index{keys!ctrl-r} \textbf{Control\_R}.
      \item \textbf{Keep queue}.
         Key: \index{keys!backslash} \textbf{backslash}.
      \item \textbf{Screenset down}.
         Key: \index{keys![} \textbf{bracketleft}.
      \item \textbf{Screenset up}.
         Key: \index{keys!]} \textbf{bracketright}.
      \item \textbf{Set playing screenset}.
         Key: \index{keys!home} \textbf{Home}.
   \end{enumber}

   Note that some of the keys have positional mnemonic value.  For example,
   for BPM control, the semicolon is at the left (down), and the apostrophe
   is at the right (up).

   Also note that the keys definable in this tab are only a subset of the
   various keys that can be used, especially keys used with the
   \texttt{Ctrl} key.

   TODO:  \index{todo!snapshot definition}
   One thing we need to figure out is just what this "snapshot"
   feature provides.
   \index{todo!keep queue}
   Another thing is the "queue" and "keep queue" features.

   \itempar{Sequence toggle keys}{keyboard!sequence toggle keys}
   Each of these keys toggles the playing/muting of one of the 32
   loop/pattern boxes.  These keys are layed out logically on the keyboard,
   and can also be shown in each loop/pattern box.  No need to list them all
   here!

   \itempar{Mute-group slots}{keyboard!mute-group slots}
   Each of these keys operates on the mute-grouping of one of the 32
   loop/pattern boxes.  These keys are layed out logically on the keyboard,
   and can also be shown in each loop/pattern box.  No need to list them all
   here!

   TODO: \index{todo!mute-group}
   One thing we need to discover is just what this mute-grouping
   means functionally.

   \itempar{Learn}{keyboard!learn}
   Learn (while pressing a mute-group key).
   This items sets the key used to initiate a learn mode.
   It is the \textbf{Insert} key by default.

   \itempar{Disable}{keyboard!disable}
   TODO: \index{todo!keyboard disable} What gets disabled?
   \index{keys!apostrophe}
   It is the \textbf{apostrophe} key by default.

   \itempar{Enable}{keyboard!enable}
   TODO: What gets enabled?
   \index{keys!igrave}
   It is the \textbf{igrave} (back-tick) key by default.

   There is much to learn about this learn/enable/disable triad!

\paragraph{Menu / File / Options / Mouse }
\label{paragraph:seq24_menu_file_options_mouse}

   This item selects the mouse-interaction method.

\begin{figure}[H]
   \centering 
   \includegraphics[scale=0.75]{menu/menu_file_options_mouse_condensed.png}
   \caption{File / Options / Mouse (Condensed View)}
   \label{fig:seq24_menu_file_options_mouse}
\end{figure}

   The default method is \textbf{seq24 (original style)}.
   The alternate method is \textbf{fruity (similar to a certain well known
   sequencer)}.

   The alternate method is presumably that of the \textsl{Fruity Loops}
   (now \textsl{FL Studio}) sequencer.  The fruity mode seems to involve the
   following (based on scanning the source code):
   
   \begin{itemize}
      \item \textbf{Left-click left side}.
         Begin a grow/shrink operation for the left side.
      \item \textbf{Left-click right side}.
         Begin a grow/shrink operation for the right side.
      \item \textbf{Left-click middle}.
         Move the object.
      \item \textbf{Left-click}.
         Add an event if nothing selected.
      \item \textbf{Middle-click}.
         Split the note?
   \end{itemize}

\paragraph{Menu / File / Options / Jack Sync }
\label{paragraph:seq24_menu_file_options_jack_sync}

   This tab sets up options for JACK synchronization.

\begin{figure}[H]
   \centering 
   \includegraphics[scale=0.75]{menu/menu_file_options_jack_sync.png}
   \caption{File / Options / Jack Sync}
   \label{fig:seq24_menu_file_options_jack_sync}
\end{figure}

   \begin{enumber}
      \item \textbf{Transport}
      \item \textbf{Jack start mode}
      \item \textbf{Connect}
      \item \textbf{Disconnect}
   \end{enumber}

   \setcounter{ItemCounter}{0}      % Reset the ItemCounter for this list.

   \itempar{Transport}{jack sync!transport}
   This items collects the following settings:

   \begin{itemize}
      \item \textbf{Jack Transport}.
         \index{JACK!transport}
         Enables synchronization with JACK Transport.
      \item \textbf{Transport Master}.
         \index{JACK!transport master}
         \textsl{Sequencer24} will attempt to serve as the JACK Master.
      \item \textbf{Master Conditional}.
         \index{JACK!master conditional}
         \textsl{Sequencer24} will fail to serve as the JACK Master if there is
         already a Master set.
   \end{itemize}

   \itempar{Transport}{jack sync!transport}
   This items collects the following settings:

   \begin{itemize}
      \item \textbf{Live Mode}.
         \index{JACK!live mode}
         Playback will be in live mode.  Use this option to allow muting and
         unmuting of patterns.
      \item \textbf{Song Mode}.
         \index{JACK!song mode}
         Playback will use only the Song Editor's data.
   \end{itemize}

   \itempar{Connect}{jack sync!connect}
   Connect to JACK Sync.

   \itempar{Disconnect}{jack sync!disconnect}
   Disconnect from JACK Sync.

\subsection{Menu / View}
\label{subsec:seq24_menu_view}

   This menu item has only one entry, \textbf{Song Editor}, 
   which is already covered by a button at the bottom of the Patterns
   window.  Selecting this item bring up the Song Editor window.
   See \figureref{fig:song_editor_window}

\subsection{Menu / Help About...}
\label{subsec:seq24_menu_about}

   This menu entry shows the "About" dialog.

\begin{figure}[H]
   \centering 
   \includegraphics[scale=0.75]{menu/menu_help_about.png}
   \caption{Help About}
   \label{fig:seq24_menu_help_about}
\end{figure}

   That dialog provides access to the credits for the program, including the
   authors and the project documentor.

\begin{figure}[H]
   \centering 
   \includegraphics[scale=0.75]{menu/menu_help_credits.png}
   \caption{Help Credits}
   \label{fig:seq24_menu_help_credits}
\end{figure}

   Shows who has worked on the program, with the original author at the top
   of the list.

\begin{figure}[H]
   \centering 
   \includegraphics[scale=0.75]{menu/menu_help_doc.png}
   \caption{Help Documentation}
   \label{fig:seq24_menu_help_doc}
\end{figure}

   Shows who has documented this project.  Say, maybe "we" can get "our"
   name there someday!  \texttt{:-)}


%-------------------------------------------------------------------------------
% vim: ts=3 sw=3 et ft=tex
%-------------------------------------------------------------------------------


% Patterns Panel

\input{seq24_patterns_panel}

% Pattern Editor

\input{seq24_pattern_editor}

% Song Editor

%-------------------------------------------------------------------------------
% seq24_song_editor
%-------------------------------------------------------------------------------
%
% \file        seq24_song_editor.tex
% \library     Documents
% \author      Chris Ahlstrom
% \date        2015-08-31
% \update      2015-09-01
% \version     $Revision$
% \license     $XPC_GPL_LICENSE$
%
%     Provides the concepts.
%
%-------------------------------------------------------------------------------

\section{Song Editor}
\label{sec:seq24_song_editor}

   The \textsl{Sequencer24 Song Editor} is used to combine all of the patterns
   into a complete tune.  It works by showing one row per
   pattern/loop/sequence in numbered columns, and the placement of each
   pattern at various musical bars in the song.

   In \textsl{Sequencer24} parlance, the Song Editor creates a
   \textsl{performance}.

\begin{figure}[H]
   \centering 
   \includegraphics[scale=0.75]{song-editor/song-editor-window.png}
   \caption{Song Editor Window}
   \label{fig:song_editor_window}
\end{figure}

   This dialog is not too complex, but
   for exposition, we break it into a top panel and the rest of the window.

\subsection{Song Editor / Top Panel}
\label{subsec:seq24_song_editor_top}

   The top panel provides quick access to song-playback actions and
   configuration.

\begin{figure}[H]
   \centering 
   \includegraphics[scale=0.75]{song-editor/song-editor-top-panel-items.png}
   \caption{Song Editor / Top Panel Items}
   \label{fig:song_editor_top_panel_items}
\end{figure}

   \begin{enumber}
      \item \textbf{Stop}
      \item \textbf{Play}
      \item \textbf{Loop}
      \item \textbf{Beats Per Bar}
      \item \textbf{Beat Unit}
      \item \textbf{Grid Snap}
      \item \textbf{Undo}
      \item \textbf{Collapse}
      \item \textbf{Expand}
      \item \textbf{Expand and copy}
   \end{enumber}

   \setcounter{ItemCounter}{0}      % Reset the ItemCounter for this list.

   \itempar{Stop}{song editor!stop}
   Stops the playback of the song.
   \index{keys!esc (stop)}
   The keystroke for stopping playback is the 'Escape' character.
   It can be configured to be another character (such as 'Space', which
   would make the space-bar toggle the playback status.

   \itempar{Play}{song editor!play}
   \index{L marker}
   Starts the playback of the song, starting at the \textbf{L marker}.
   \index{keys!space (play)}
   The keystroke for starting playback is the 'Space' character.

   \itempar{Loop}{song editor!loop}
   \index{L marker}
   \index{R marker}
   Play the song, looped between the \textbf{L marker} and the
   \textbf{R marker}.
   This button is a state button, and its appearance indicates when it is
   depressed, and thus active.
   If this button is deactivated during playback, then playback will
   continue past the \textbf{R marker}.

   \itempar{Beats Per Bar}{song editor!beats/bar}
   Part of the time signature, and specifies the number of beat units per bar.
   The possible values range from 1 to 16.

   \itempar{Beat Unit}{song editor!beat unit}
   Part of the time signature, and specifies the size of the beat unit:
   1 for whole notes; 2 for half notes; 4 for quarter notes; 8 for eight notes;
   and 16 for sixteenth notes.

   \itempar{Grid Snap}{song editor!grid snap}
   Grid snap selects where the patterns will be drawn.
   Unlike the \textbf{Grid Snap} of the Pattern Editor, the units
   of the Song Editor snap value are in fractions of a measure length.
   The following values are supported:
   1/1, 1/2, 1/4, 1/8, 1/16, and 1/32.

   \itempar{Undo}{song editor!undo}
   The Undo button will roll back the last change in the layout of a
   pattern.  Each time it is clicked, the most recent change will be undone.
   It will roll back one change each time it is pressed.
   It is not certain what the undo limit is, however.
   There is no Redo button in the Song Editor.

   \itempar{Collapse}{song editor!collapse}
   This button collapses the song between the \textbf{L marker} and the
   \textbf{R marker}.
   What this means is that, if there is song material (patterns) before the
   \textbf{L marker} and after the \textbf{R marker},
   and the \textbf{Collapse} button is
   pressed, any song material between the L and R markers is wiped out, and
   the song material after the \textbf{R marker} is moved leftward to
   the \textbf{L marker}.

   Collapsing occurs in all tracks present in the Song Editor.

   \itempar{Expand}{song editor!expand}
   This button expands the song between the
   \textbf{L marker} and the \textbf{R marker}.
   It inserts blank space between these markers, moving the song material
   that is after the \textbf{R marker}
   to the right by the duration of the blank space.

   Expansion occurs in all tracks present in the Song Editor.

   \itempar{Expand and copy}{song editor!expand and copy}
   This button expands the song between the \textbf{L marker} and the
   \textbf{R marker} much like the \textbf{Expand} button.
   However, it also copies the original data that is present after the
   \textbf{R marker}, and pastes it into the newly-available space between
   the L and R markers.

\subsection{Song Editor / Arrangement Panel}
\label{subsec:seq24_song_editor_arrangement_panel}

   The arrangement panel is the middle section shown in
   \figureref{fig:song_editor_window}.  It is also known as the
   "piano roll" of the song editor. Here, we zero in on its many
   features.

   The following figure is taken from a conventional MIDI file, imported,
   with a few long tracks, rather than a large number of smaller patterns.
   In other words, the patterns used here are very long, and used only once
   in the song.
   
   We might need to provide an example that shows off \textsl{Sequencer24}'s
   pattern features better, at some point.

   Please note that, if playback is started with the Song Editor as the
   active window, then the pattern boxes in the patterns panel will
   show as armed/unarmed (unmuted/muted) depending upon whether or not the
   pattern is shown as playing (or not) at the current playback position in
   the Song Editor piano roll.

\begin{figure}[H]
   \centering 
   \includegraphics[scale=0.75]{song-editor/song-editor-window-full-items.png}
   \caption{Song Editor Arrangement Panel, Annotated}
   \label{fig:song_editor_window_full_items}
\end{figure}

   \index{measures ruler}
   It consists of a \textsl{measures ruler} (bar indicator) at the top, a
   numbered patterns column at the left with a muting indicator, and the
   grid or roll section.  There are a lot of hidden details in the
   arrangement panel, as the figure shows.  Here are the main sections we
   will deal with:

   \begin{enumber}
      \item \textbf{Patterns Column}
      \item \textbf{Piano Roll}
      \item \textbf{Measures Ruler}
   \end{enumber}

   These items are discussed in the following sections.

\subsubsection{Song Editor / Arrangement Panel / Patterns Column}
\label{subsubsec:seq24_song_editor_arrangement_panel_patterns_column}

   Here are the items to note in the patterns column:

   \begin{enumber}
      \item \textbf{Number}.
         Not yet sure what the number on the left means.
         The number of the screen set?
      \item \textbf{Title}.
         \index{pattern!title}
         \index{pattern!name}
         The title is the name of the pattern, for easy reference.
      \item \textbf{Channel}.
         \index{pattern!channel}
         The channel number appears at the right of the title.
      \item \textbf{Buss-Channel}.
         \index{pattern!span}
         This pair of numbers shows the MIDI buss number used in the pattern and
         the channel used for the pattern.
      \item \textbf{Beat/Measure}.
         \index{pattern!beat}
         This pair of numbers is the standard time-signature of the pattern.
      \item \textbf{Mute Indicator}.
         \index{song editor!mute indicator}
         The letter M is in a black box if the track/pattern is muted, and a
         white box if it is unmuted.
      \item \textbf{Empty}.
         Empty tracks are indicated by a dark-gray filling.
   \end{enumber}

   The patterns column shows a list of all of the patterns that have been
   created in the current song.  Each pattern in this list has a track of
   pattern layouts associated with it in the piano roll section.

   \index{patterns column!left click}
   Left-clicking on the pattern name or the "M" button toggles the muting
   status of the track.

   \index{patterns column!right click}
   Right-clicking on the pattern name or the "M" button brings up the same
   pattern editing menu as discussed in
   \sectionref{subsubsec:seq24_patterns_pattern_filled}.
   Recall that this context menu has the following entries:
   Edit..., Cut, Copy, Song, and Midi Bus.

\subsubsection{Song Editor / Arrangement Panel / Piano Roll}
\label{subsubsec:seq24_song_editor_arrangement_panel_roll}

   The "Piano Roll" section of the arrangement panel is where patterns or
   subsections are inserted, deleted, shrunk, lengthened, or moved.

   Here are the items to note in the Piano Roll area:

   \begin{enumber}
      \item \textbf{Single}.
         In the diagram, under the word "Single", is a very small pattern.
         It is small because it consists only of some MIDI Program Change
         messages meant to set the programs on a Yamaha PSS-790 keyboard.
      \item \textbf{Multiple}.
         This items is the same pattern as in "Single", but dragged out for
         multiple repetitions, simply to show how even the shortest patterns
         can be easily replicated.
      \item \textbf{Pattern Subsection}.
         \index{song editor!middle click}
         \index{pattern subsection}
         Middle-clicking inside a pattern inserts a selection position
         marker in it, breaking the pattern into two equal pieces.
         We call each piece a \textsl{pattern subsection}.
         This division can be done over and over.
         (Note that, in the Song Editor, a middle-click
          \textsl{cannot} be simulated by ctrl-left-click.)
      \item \textbf{Selection Position}.
         A selection position is a marker that divides a pattern into two
         pieces, called \textsl{pattern subsections}.  This makes it easy to
         select smaller portions of a pattern for editing or deleting.  It
         is especially useful for making holes in a pattern.  There may be
         other uses of a selection position that we have not yet discovered.
      \item \textbf{Selection}.
         By clicking inside a pattern or a pattern subsection, it darkens to
         denote that it is selected.
         A pattern subsection can be deleted by the
         \index{keys!delete}
         Delete key, copied by the
         \index{keys!ctrl-c}
         \texttt{Ctrl-C} key, and then inserted (one or more times) by the
         \index{keys!ctrl-v}
         \texttt{Ctrl-V} key.  When inserted, each insert goes immediately
         after the current item or the previous insertion.  The same can be
         done for whole patterns.
      \item \textbf{Section Length}.
         Looking closely at the diagram where the arrows point, small
         squares in the corner of the patterns can be seen.  By grabbing
         that square with a left-click, the square can be moved horizontally
         to either lengthen or shorted the pattern or pattern subsection, if
         there is room to move in the desired direction.
         It doesn't matter if the item is selected or not.
      \item \textbf{Section Movement}.
         If, instead of grabbing the section length handle, one grabs inside
         the pattern or pattern subsection, that item can be moved
         horizontally, as long as their is room.  Or course, left-clicking
         inside the item will also cause it to show as selected.
      \item \textbf{Expansion}.
         Originally, all the long patterns of this song were continuous.
         But, by setting the L and R markers, and using the \textbf{Expand}
         button, we opened up some silent space in the song.
   \end{enumber}

   The \textsl{Sequencer24} help files refer to work in the Song Editor as the
   "Performance Edit" or "Performance Mode".  Adding a pattern in this
   window is a bit like adding a note in the Pattern Editor.
   One clicks, holds, and drags the mouse to insert a copy of the pattern
   associated with the row in which one is dragging.  The longer one drags,
   the more copies of the pattern that are inserted.

   \index{song editor!right click hold}
	Right-click on the arrangement panel (roll) to enter
   draw mode, and hold the button.

   \index{song editor!left click right hold}
   Then left-click the mouse to insert one copy of the pattern.  The
   inserted pattern will show up as a box with a tiny representation of the
   notes visible inside.  (Some patterns, however, can be less than a
   measure in length, resulting in a tiny box.)

   \index{song editor!right left hold drag}
   To keep adding more copies of the pattern, continue to hold both buttons
   and drag the mouse rightward.

   \index{song editor!middle click}
   Middle-click on a pattern to drop a new selection position into the
   pattern,
   \index{song editor!pattern subsection}
   which breaks the pattern into two equal \textsl{pattern subsections}.
   Each middle-click on the pattern adds a new selection position,
   halving the size of the subsections as more pattern subsections are
   added.

   \index{song editor!left click}
   When a pattern or a pattern subsection is left-clicked, it is marked as
   dark gray.
   \index{song editor!right left hold drag}
   When a right-left-hold-drag action is done in this gray area, the result
   is to \textsl{delete} that pattern section or subsection.
   \index{keys!delete}
   One can also hit the Delete key to \textsl{delete} that pattern section
   or subsection.

\subsubsection{Song Editor / Arrangement Panel / Measures Ruler}
\label{subsubsec:seq24_song_editor_arrangement_panel_measures_ruler}

   The \textsl{measures ruler} is the ruled and numbered section at the top
   of the arrangement panel.  It provides a place to put the left and right
   markers.  In the \textsl{Sequencer24} documentation, it is called the "bar
   indicator".

   \index{measures ruler!left-click}
   Left-click in the measures ruler to drop an
   \index{L anchor}
   \textbf{L anchor} on the measures ruler.
   \index{measures ruler!right-click}
   Right-click in the measures ruler to drop an
   \index{R anchor}
   \textbf{R anchor} on the measures ruler.

   Once these anchors are in place, then use
	the \textsl{Collapse} and \textsl{Expand} buttons to modify the
   placement of the pattern events.

   Note that the \textbf{L marker} serves as the start position for playback
   in the Song Editor.  One can change the start position only when the
   performance is not playing.

%-------------------------------------------------------------------------------
% vim: ts=3 sw=3 et ft=tex
%-------------------------------------------------------------------------------


% Configuration file

\input{seq24_rc_file}

% User file

\input{seq24_usr_file}

% Man page

%-------------------------------------------------------------------------------
% seq24_manpage
%-------------------------------------------------------------------------------
%
% \file        seq24_manpage.tex
% \library     Documents
% \author      Chris Ahlstrom
% \date        2015-08-31
% \update      2015-08-31
% \version     $Revision$
% \license     $XPC_GPL_LICENSE$
%
%     Provides the man page section of seq24-user-manual.tex.
%
%-------------------------------------------------------------------------------

\section{Sequencer24 Man Page}
\label{sec:seq24_man_page}

   This section presents the contents of the \textsl{Sequencer24} man page, but
   not exactly in \textsl{man} format.  Also, an item or two are shown that
   somehow didn't make it into the man page, and minor corrections and
   formatting tweaks were made.

   \textsl{Sequencer24} - Real time MIDI sequencer.

   \textsl{Sequencer24} is a real-time MIDI sequencer. It was created to provide a
   very simple interface for editing and playing MIDI 'loops'.

   \begin{verbatim}
       seq24 [OPTIONS] [FILENAME]
   \end{verbatim}

   \textsl{Sequencer24} accepts the following options, plus an optional name of a
   MIDI file.

   \setcounter{ItemCounter}{0}      % Reset the ItemCounter for this list.

   \optionpar{-h}{--help}
      Display a list of all command-line options.

   \optionpar{-v}{--version}
      Display the program version.

   \optionpar{N/A}{--file [filename]}
      Load MIDI file on startup.
      \textbf{Bug:}
      \index{bugs!--file option doesn't exist}
      This option does not exist.

   \optionpar{-m}{--manual\_alsa\_ports}
      \textsl{Sequencer24} won't attach ALSA ports.

   \optionpar{-s}{--showmidi}
      Dumps incoming MIDI to the screen.

   \optionpar{-p}{--priority}
      Runs at higher priority with FIFO scheduler.

   \optionpar{N/A}{--pass\_sysex}
      Passes any incoming SYSEX messages to all outputs.

   \optionpar{-i}{--ignore [number]}
      Ignore ALSA device [number].

   \optionpar{-k}{--show\_keys}
      Prints pressed key value.

   \optionpar{-x}{--interaction\_method [number]}
      Select the mouse interaction method.
      0 = seq24 (the default); and 1 = fruity loops method>

   \optionpar{-j}{--jack\_transport}
      \textsl{Sequencer24} will sync to JACK transport.

   \optionpar{-J}{--jack\_master}
      \textsl{Sequencer24} will try to be JACK master

   \optionpar{-C}{--jack\_master\_cond}
      JACK master will fail if there is already a master

   \optionpar{-M}{--jack\_start\_mode [x]}
      When \textsl{Sequencer24} is synced to JACK, the following play modes are
      available: 0 = live mode; and 1 = song mode, the default.

   \optionpar{-S}{--stats}
      Print statistics on the command-line while running.

   \optionpar{-U}{--jack\_session\_uuid [uuid]}
      Set the UUID for the JACK session.

   \texttt{\$HOME/.seq24rc} stores the user settings for \textsl{Sequencer24}.

   The old project homepage is at
   \url{http://www.filter24.org/seq24/} the new
   one is at \url{https://edge.launchpad.net/seq24/}.
   It is released under the GNU GPL license.

   \textsl{Sequencer24} was written by Rob C. Buse \url{mailto:seq24@filter24.org}
   and the \textsl{Sequencer24} team.

   This manual page was written by Dana Olson
   \url{mailto:seq24@ubuntustudio.com} with additions from Guido Scholz
   \url{mailto:guido.scholz@bayernline.de}.

   \begin{verbatim}
Version 0.9.2                   November 20 2010                       Sequencer24(1)
   \end{verbatim}

%-------------------------------------------------------------------------------
% vim: ts=3 sw=3 et ft=tex
%-------------------------------------------------------------------------------


% Building and debugging Seq24

% \input{yum_build}

\section{Summary}
\label{sec:summary}

   In summary, we can say that you will find \textsl{Seq24} intriguing.

   There are some topics that this document does not yet treat ...:

   Contact: If you have ideas about \textsl{Seq24} or a bug report, please
   email us (at \url{mailto:seq24@filter24.org}).
   If it's a bug report, please add \textbf{[BUG]} to the Subject.

% References

\input{seq24_references}

\printindex

\end{document}

%-------------------------------------------------------------------------------
% vim: ts=3 sw=3 et ft=tex
%-------------------------------------------------------------------------------
